\documentclass[12pt]{article}
\usepackage[latin1]{inputenc}
\usepackage{graphicx,subfigure}
\topmargin -1mm
\oddsidemargin -1mm
\evensidemargin -1mm
\textwidth 165mm
\textheight 220mm
\parskip 2mm
\parindent 3mm
%\pagestyle{empty}

\begin{document}



\begin{center}

{\Large \bf VCLab the Guide.}

\bigskip

written by Xing Wang

\bigskip

Central South University, China

\bigskip

Version \today

\end{center}


\bigskip

{\em This is part of the documentation of the Virtual CALPHAD Laboratory
software.}

\newpage

\tableofcontents

\newpage

\section{Introduction}
VCLab is a open source package for CALPHAD calculations. 

The VCLab guide is mainly explain the files and control flags implemented in the code. This guide also tries to give an impression, how VCLab works. The guide continues to grow as new features are added to the code. It is therefore useful to check the online version of the VCLab guide from time to time, to learn about new features added to the code.



\section{Installation}




\section{Input files}
The VCLInput.txt is the central input file of VCLab. It determines what to do and how to do it.
One example for this file list below:

\begin{table}[!hbp]
\begin{tabular}{l l l }%��ʼ���Ʊ���
Deminsion             & = & 1D !\\
Mode                  & = & Equilibrium \\
Database File         & = & AlZn.TDB! \\
Elements		      & = & Al, Zn!\\
Compositions	      & = & 0.6, 0.4!\\
Pressure	          & = & 10500!\\
Varibles		      & = & Temperature !\\
V Start		          & = & 300 !\\
V End		          & = & 2000 !\\
V Interval	          & = & 10 !\\
Result Output File	  & = & Result.txt !\\
Global Grid Interval  & = & 0.05 !\\
\end{tabular}
\end{table}

Each line consists of a keywrod (i.e. a string) the equation sign "=" and one or several values. Values are separated by a colon ",". One parameter one single line ended by !. Comments are normally preceded by the number sign \#. Defaults are supplied for some parameters. For example, a default for the "Global Grid Interval" is 0.5.

The following sections will describe the parameters given in the VCLInput file.

\subsection{Deminsion}
The tag Deminsion determines the number of variables. $n$D means n variables. $0$D corresponding single point calculation in Thermo-Calc, $1$D is step, $2$D is map. 


\subsection{Mode}
\subsection{Database File}
\subsection{Varibles}
\subsection{Global Grid Interval}


\section{Point calculations}


\section{Line calculations}


\section{Gibbs energy surface}

\section{Algorithm}

The algorithm to calculate thermodynamic equilibria for
multi-component systems was widely introduced in varies papers
and books. In this document, the algorithm is explained in the
way that directly relate to programming.
\subsection{Gibbs energy funciton}
In thermodynamics, phases equilibrium is a state when the chemical potential
in all phases are equal for every component at given conditions. In mathematics, phases
equilibrium is the global minimum point of the overall Gibbs energy function,
with given constraints.

The Gibbs energy function can be written as:
\begin{equation}
G = \sum_{\alpha} n^{\alpha} G^{\alpha}(T,P,y^{\alpha}_{ks}) \label{eq:gsum}
\end{equation}
where $n^{\alpha}$ is the moles amount of the phase $\alpha$ and
$G^{\alpha}$ is the Gibbs energy per mole formula of phase $\alpha$. T and P
is the temperature and pressure, $y^{\alpha}_{ks}$ is the site fractions in the
 s lattices of phase $\alpha$.

\noindent
The equality constraint are:

(1) sum of the site fractions on sublattice s of phase $\alpha$ should be unity:
\begin{equation}
\sum_k y^{\alpha}_{ks} - 1= 0
\end{equation}

(2) sum of the components amount should equal to overall components amount:
\begin{equation}
\sum_{\alpha}n^{\alpha}N^{\alpha}_{A} - \tilde N_{A}=0
\end{equation}

\noindent
where ${\tilde N}_{A}$ is the overall amount of component A.



The inequality constraint are:
\begin{equation}
\ y^{\alpha}_{ks} > 0
\end{equation}

\begin{equation}
\ n^{\alpha} > 0
\end{equation}

\noindent
charge balance is not consider at the moment.


\subsection{Lagrangian method}
A strategy for finding the minima of a function subject to equality
constraints is lagrangian methods. Lagrange multipliers can be used
to convert it into an unconstrained problem whose number of variables
is the original number of variables plus the original number of
equality constraints.

\begin{equation}
L = G + \sum_{\alpha} \lambda_s^{\alpha} (\sum_k y^{s, \alpha}_{k} - 1)
+ \sum_{A} \mu_{A}(\sum_{\alpha}n^{\alpha}N^{\alpha}_{A} - \tilde N_{A})
\end{equation}
\noindent
where $\lambda^{\alpha}, \mu_{\rm A}$ are multipliers.

The minimum can be reach, when the derivative of the Lagrangian with
respect to all the variables are zeros.

\begin{equation}
G^{\alpha}_M + \sum_{A} \mu_{A} N^{\alpha}_{A} = 0
\end{equation}

\begin{equation}
\sum_k y^{s, \alpha}_{k} - 1= 0
\end{equation}

\begin{equation}
n^{\alpha}\frac{\partial G^{\alpha}}{\partial y^{s, \alpha}_{k}}
- n^{\alpha}\mu_{A}\frac{\partial N^{\alpha}_{A}}{\partial y^{s, \alpha}_{k}}
- \lambda_s^{\alpha} = 0
\end{equation}

\begin{equation}
\sum_{\alpha}n^{\alpha}N^{\alpha}_{A} - \tilde N_{A}=0
\end{equation}

In order to solve this nonlinear equations, Newton's method can be used.
We then get:


\refstepcounter{equation}\label{eq:HillertMatrix}

\[
\left(
\begin{tabular}{cccccc}
 0 & 1 & 0 & 0 & 0 &0\\
\\
1 & $\frac{\partial^2 G^{\alpha}}{\partial y_k^{s, \alpha}\partial y_{k^{\prime}}^{s, \alpha}} $ &
0 & $\frac{\partial N^{\alpha}_{A}}{\partial y^{s, \alpha}_{k}}$
& $\frac{\partial^2 G^{\alpha}}{\partial y_k^{s, \alpha} \partial T}$
& $\frac{\partial^2 G^{\alpha}}{\partial y_k^{s, \alpha} \partial P}$ \\
\\
0 & $\frac{\partial G^{\alpha}}{\partial y_k^{s, \alpha}} +
\mu_{A}\frac{\partial N^{\alpha}_{A}}{\partial y^{s, \alpha}_{k}}$
& 0 & $N^{\alpha}_{A}$
& $\frac{\partial G^{\alpha}}{\partial T}$
& $\frac{\partial G^{\alpha}}{\partial T}$  \\
\\
 0 & $n^{\alpha}\frac{\partial N^{\alpha}_{A}}{\partial y^{s, \alpha}_{k}}$
 & $N^{\alpha}_{A}$ & 0 & 0 & 0\\
\end{tabular}
\right)
\left(
\begin{tabular}{c}

$\Delta \lambda_s^{\alpha}$\\
\\
$\Delta y^{s, \alpha}_{k}$\\
\\
$\Delta n^{\alpha}$\\
\\
$\Delta \mu_{A}$\\
\\
$\Delta T$\\
\\
$\Delta P$\\
\end{tabular}
\right)
=
\left(
\begin{tabular}{cccccc}
$\sum_k y^{s, \alpha}_{k} - 1$\\
\\
$ \frac{\partial G^{\alpha}}{\partial y^{s, \alpha}_{k}}
- \mu_{A}\frac{\partial N^{\alpha}_{A}}{\partial y^{s, \alpha}_{k}}
- \lambda_s^{\alpha}$\\
\\
$G^{\alpha}_M + \sum_{A} \mu_{A} N^{\alpha}_{A}$\\
\\
$\sum_{\alpha}n^{\alpha}N^{\alpha}_{A} - \tilde N_{A}$\\
\end{tabular}
\right)
\\
\\ (\ref{eq:HillertMatrix})
\]

We called left coefficient matrix the equilibrium matrix.
We begin with a first guess of the variables $X_0$, then solve this
equations to get new values of the variables $X_1$ as:
\begin{equation}
X_1 = X_0 + \Delta X
\end{equation}
The process is repeated, until the changes are sufficiently small.

\subsection{External conditions}

There are more variables than equations in eq.~\ref{eq:HillertMatrix}. Therefore, in
order to solve the equations, some of the variables must be fixed. That exactly
external conditions means.

When some variables are fixed, the corresponding columns and rows in the euqilibrium matrix
should be removed. For example, when T and P fixed.

\refstepcounter{equation}\label{eq:fixed T and P}

\[
\left(
\begin{tabular}{cccc}
 0 & 1 & 0 &0\\
\\
1 & $\frac{\partial^2 G^{\alpha}}{\partial y_k^{s, \alpha}\partial y_{k^{\prime}}^{s, \alpha}} $ &
0 & $\frac{\partial N^{\alpha}_{A}}{\partial y^{s, \alpha}_{k}}$ \\
\\
0 & $\frac{\partial G^{\alpha}}{\partial y_k^{s, \alpha}} +
\mu_{A}\frac{\partial N^{\alpha}_{A}}{\partial y^{s, \alpha}_{k}}$
& 0 & $N^{\alpha}_{A}$ \\
\\
 0 & $n^{\alpha}\frac{\partial N^{\alpha}_{A}}{\partial y^{s, \alpha}_{k}}$
 & $N^{\alpha}_{A}$ & 0 \\
\end{tabular}
\right)
\left(
\begin{tabular}{c}

$\Delta \lambda_s^{\alpha}$\\
\\
$\Delta y^{s, \alpha}_{k}$\\
\\
$\Delta n^{\alpha}$\\
\\
$\Delta \mu_{A}$\\
\\
\end{tabular}
\right)
=
\left(
\begin{tabular}{c}
$\sum_k y^{s, \alpha}_{k} - 1$\\
\\
$ \frac{\partial G^{\alpha}}{\partial y^{s, \alpha}_{k}}
- \mu_{A}\frac{\partial N^{\alpha}_{A}}{\partial y^{s, \alpha}_{k}}
- \lambda_s^{\alpha}$\\
\\
$G^{\alpha}_M + \sum_{A} \mu_{A} N^{\alpha}_{A}$\\
\\
$\sum_{\alpha}n^{\alpha}N^{\alpha}_{A} - \tilde N_{A}$\\
\end{tabular}
\right)
\\
\\ (\ref{eq:fixed T and P})
\]



\subsection{Global minimum}

Four kinds of results we may get from eq.~\ref{eq:HillertMatrix}when
different initial values are given:

(1) divergency

(2) local minimum

(3) global minimum

(4) wrong minimum out of bounds (inequality constraint)

That's the reasons why we should use a global minimum technology
to get a reasonable initial value.

In most papers and book, local minimum is believed to be related with miscibility gap.
Because the Gibbs energy of phase with miscibility gap is not a convex function. Actually,
local minimum is exist for all case. In eq.~\ref{eq:gsum}, the Gibbs energy of all
phases are convex function can not guaranteed their sum is a convex function.
Because new variables $n^{\alpha}$ are add.

\section{Summary}

Not finished yet. More detail should be added.

\end{document}

